Previous Work (evtl. mit in die Introduction) / Related Work

\begin{itemize}
\item Bezug nehmen auf
\item DeepDose
\item Weitere Paper zu dem Thema (Protonen mit LSTM, weitere Dosisberechnungen, siehe Paper, dass Christian gesendet hat)
\end{itemize}

Radiotherapy: 

\textbf{https://iopscience.iop.org/article/10.1088/1361-6560/ab7630}\break
using 3d unet architecture and patch based approach to estimate dose in prostate cancer patients for individual segments of an entire treatmentplan. input data fro unet include patient information in the form of ct, as well as information form the accellerator such as beam angle and beam shape. they achieve very good results with 99.9 ± 0.3 for 3\%/3mm gamma passrate. and 1 minute for dose estimation for one patient in total

\textbf{https://aapm.onlinelibrary.wiley.com/doi/full/10.1002/mp.14658}\break
dose prediction for protonfields in highly heterogenous tissues using long short term memory (LSTM) networks. it treats the affected part of the tissue as a sequence of 2 inputs and calculates the respective dose in the scheme of a marching beam trough the tissue. the reach robustnes by generating artificial phantom cases for the dose predictions. results show 98.57\% gamma passrate (1\%/3mm) for artificial cases and an average gama passrate of 97.85\% for patient test cases.

\textbf{https://www.ncbi.nlm.nih.gov/pmc/articles/PMC7870566/}\break
dose estimation for segments was achieved by combining fluence maps from segments aswell as patient anatomies. they generated 3d fluence volumes from 2d fluence maps including information of beam angle. an 3d unet like architecture with residual blocks inside the convolutional building block was used for prediction. the overall dose variation normalized to the prescribed dose was 0.17\% ± 2.28\%. 

\textbf{High-Particle Simulation of Monte-Carlo Dose Distribution with 3D ConvLSTMs}\break
to achieve a precise dose distribution in the desired 3d volume they use an approach utilizing active denoising of monte carlo simulations with deep learning. they implemented a 3d LSTM architecture inside the skipconnection of a 3d unet architecture. a sequence of noisy monte carlo simulations was used as the input and the output is the denoised doe prediction. the loss function was constructed from the L1 loss and the structural similarity index measure. gamma pass rate for 5 patients was 94.1\% ± 1.2\% and the L1 was was 4$\cdot$10\textsuperscript{-3} ± 1$\cdot$10\textsuperscript{-3}.

\textbf{DeepMCDose: A Deep Learning Method for Efficient Monte Carlo Beamlet Dose Calculation by Predictive Denoising in MR-Guided Radiotherapy}
the proposed network consists of 3 seperate Unets to seperately analyse the 3 input channels consisting of unsersampled MC dose, MC x-ray fluence and the ct geometry for a single beamlet. the output from each network is then combined to calculate the residual needed dose to achieve the denoised dose distribution. the normalized mean absolute error reduced from 25.7\% on the undersamplet MC simulation to 0.106\% for the network output. The simulation of 380s for an fully sampled beamlet was reduced to 220 ms for the network prediction

https://www.ncbi.nlm.nih.gov/pmc/articles/PMC7115345/