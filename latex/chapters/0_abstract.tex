
\textbf{Purpose:} Monte Carlo (MC) simulations are time intensive and therefore not applicable for real-time dose calculations without major simplifications.  Recent research efforts aim at the use of Deep Learning (DL) tools for dose calculation. 
However, the accuracy of pretrained DL models remains unclear if conditions between training data and applied data change.
Different conditions may include tumor entity, source-to-surface distance (SSD), field size and shape, gantry angle and tissue density. In this work, we therefore developed a DL dose prediction model and investigated the accuracy and robustness of dose prediction on unseen data.\\

\textbf{Methods:} A DL framework was developed to predict dose distributions based on patient CT and irradiation field information. 
Two training datasets were defined based on clinical MR-Linac treatment plans of (A) 40 primary prostate cancer patients and (B) 40 patients with either prostate, head and neck, breast or liver cancers. 
Both training sets were composed of approximately 2000 individual segments from different angles. 
Gold standard dose distributions, used as the target for model training and testing, were obtained by segment-wise MC dose simulation using a dedicated EGSnrc MR-Linac model. 
The training datasets were used to train two separate 3D-UNet for dose prediction. 
For evaluation, both trained models were applied to data representing three different conditions: (1)~A~set of 5 unseen prostate plans, (2) 5 head and neck, breast and liver plans each and (3) 15 lymphnode plans, for which the conditions were unseen by both models. 
The DL dose predictions were compared against gold standard using gamma analysis (3mm/3\%, 10\% cutoff) and evaluated by Wilcoxon signed-rank test.\\

\textbf{Results:} Both DL models were successfully trained and allowed for segment-wise dose prediction.
The network trained on training dataset (A) reached mean gamma passrates of  99.1\% ± 1.3\% when tested against (1).
Accuracy decreased significantly when tested against (2) and (3) with mean gamma passrates of 89.9\%~±~4.1\%, 67.4\%~±~5.3\%, 77.4\%~±~10.2\%, 93.0\%~±~6.1\% for liver, breast, head and neck and lymphnode plan segments respectively.
In comparison the mixed model (B) applied to the test datasets (1), (2) and (3) yielded mean gamma pass rates of 99.2\%~±~1.0\%, 97.6\%~±~1.0\%, 83.6\%~±~4.6\%, 88.8\%~±~8.6\% and 96.0\%~±~5.8\% for prostate, liver, breast, head and neck and lymphnode plans respectively. 