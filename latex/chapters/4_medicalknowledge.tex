\subsection{Tumor Radiation}

As mentioned in the previous section external tumor radiation is a widespread non-invasive modality for cancer treatment with or without tumor resection.
Studies indicate that it should be used in over 50\% of all cancer cases once in the course of the illness~\cite{delaney_role_2005}. 
Particles used in radiation therapy interact with the patient's tissue in a way that causes direct or indirect biological damage to cells.
Radiated particles release free radicals inside the tissue, which cause \ac{DNA} damage in the form of single- or double strand breaks, leading to the induction of apoptosis~\cite{kaina_dna_2003} and therefore the destruction of the radiated cell.
Cells off healthy tissue are able to repair such \acs{DNA} damage under normal conditions. 
Cancerous cells repair capabilities are downregulated due to the microenviromental conditions, such as hypoxia, low pH and nutrient deficiency, in the primary cancer region~\cite{li_dna_2021}.
Radiation takes advantage of these down-regulated repair mechanisms, resulting in the destruction of cancer cells while limiting radiation to a point where healthy tissue can regenerate.\\
Photons attenuation is described by an exponential decay, meaning that a photon beam can not be entirely shielded.
This is of crucial importance for the optimal planning of radio therapy.
Since tumors are usually located in deeper regions of the body (see the example of prostate cancer in \autoref{fig:prostate_oar}), it is inevitable to irradiate healthy tissue that is located in front of or behind the tumor volume with respect to the radiation direction. 
Aim of a treatment plan optimization process is to achieve an optimal dose coverage over the entire target volume, while minimizing the radation dose in surrounding \acs{OAR}.
Speficic limits for dose coverage of tumor volumes as well as \acs{OAR} are set by the intitutional standard operating procedure, which accounts for recent scientific findings in literature.
First step during treatment planning is the definition of an optimal dose distribution over the entire radiated volume.
In this first step, the planning software tries to find a dose distribution that satisfies all the given margins and limits, without considering the actual achievable dose distribution with respect to machine parameters and settings.
To achieve this first optimal dose distribution the planning software takes the tumor volume and all delineated \acs{OAR} with their defined dose limits in consideration.
In the next step the treatment plan is created in an iterative manner trying to achieve the given optimal dose distribution using actual possible accelerator settings.
The result from the planning process are multiple segments consisting of a \ac{MLC} configuration, accelerator gantry angle and a defined amount of particles radiated.
This created plan offers an optimal dose coverage of the target volume and sparing of \ac{OAR} according to the institutional standard operating procedure.\\
Further decrease of bad side effects in radiated healthy tissues is achieved by treatment fractionation.
By dividing the treatment into several sessions with lower radiation doses, the increased repair capabilities of healthy tissue are utilized.
Time between and dose for fractions is chosen according to findings in current literature yielding information about repair capabilities of cancerous and healthy cells.
This leads to a steady decrease in cancer cell population in the radiated region.

\begin{figure}
    \centering
    \includegraphics[width=0.5\textwidth]{example_image.pdf}
    \caption{
        CT image of a prostate caner patient including the dose distribution of the respective treatment plan. Areas of high dose deposition are displayed in red and orange. Left and right femur head aswell as rectum displayed in blue and orange and red respectively are displayed along the \acs{PTV} displayed in green.
        }\label{fig:prostate_oar}
\end{figure}

\subsection{MR Linear Accelerator}

In the current state of the MR-Linac emplyment the use of a conventional \ac{CT} as the base for treatment planning is inevitable.
On the base of this \acs{CT} target volume and \ac{OAR} are defined by an oncologist.
In an conventional setting this delineation and the following treatment plan on the base of this first \ac{CT} is used for all treatment fractions.
This results in uncertainties during treatment because it is only assumed that the position of organs remains constant during the entire course of illness.
Especially for moving \acs{OAR} such as bldder or sigmoid in the case of a lower abdomen cancer patient this poses a problem becuase it can not be assured that they did not move during treatment of between fractions without a control \acs{CT}.
The MR-Linac offers the aquisition of an MR image before treatment.
With this control MR image the attending oncologist can verify that previous delineations are still valid and if they are not valid, the MR-Linac offers a solution to adapt the treatment plan to interfraction movement of target volume or \acs{OAR}.
\acs{OAR} are registered from the CT onto the MR image and the treatment plan can be adapted to change in position or shape.
Future studies are ongoing investigating if this treatment modality leads to a decrease in safety margins accounting for unvertainties regarding patient positioning aswell as tissue movement.\\
Current research is going towards the further exploration of opportunities involving the MR-Linac.
The vision behind the MR-Linac would be to adapt the treatment plan based on real-time image aquisition during the treatment.
Critical steps to enable this proposed workflow are the optimization of all involved processes of the treatment planning pipeline to be applicable in real-time and to be based on MR images only.
An MR-only worklow is promised to ultimately resulting in in-time adaption to intra-fraction movements such as unwanted patient movement aswell as breathing motion potentially leading to a significant decrease in safety margins.
All neccecary steps to be automated or optimized would be the delineation of tumor volumes and \acs{OAR} on MR images, treatment planning and optimization aswell as pseudo-CT creation from MR images to enable \acs{DD} calculations.
Pseudo-CT creation is needed for \acs{DD} because \acs{MRI} is not a quantitative imaging modality, meaning that the pixel values of \acs{MRI} images are not correlated to the electron density of the tissue represented by that tissue, which are needed for dose calculation algorithms.
Therefore a translation from qualitative MRI data to quantitative electron density imformation is needed to estimate deposited dose inside a a given volume or patients anatomy.\\
\acs{DD} is the part of the MR-only treatment pipeline we are focusing on in this study, trying to optimize its computational recources aswell as computation time. 
As previoulsy mntioned \acs{MC} simulation is used for the simulation of particle histories to give an estimate of deposited dose inside the patients anatomy.
These \acs{MC} simulations are used in the secondary dose calculation for treatment plan optimization aswell as the dose verification.

\subsection{Physical Fundamentals}

Photons radiated from the linear accelerator interact with the patients tissue.
All interaction processes are of stochastic nature, meaning it can not exactly determined when a particle will interact in which manner during is path trough a dense volume.
Interaction processes of photons aswell as electrons are very well quantitatively described in literature giving probability distributions for different energies aswell as densities of tissues.
Main interaction processes of photons in the energy sprectrum of the MR-Linac are elastic aswell as inelastic interactions such as rayleigh scattering, the photo effect or compton scattering, which leads to small changes in trajectory of photons inside a medium.
With increased depth inside tissue this change in trajecotry leads to an widening of the initial beam.
During the latter two interaction processes photons transmitt all of their or partial energy onto secondary electrons.
Shorter mean free path lengths and higher ionization capabilities of electrons lead to dose deposition in the nearby surrounding tissue, which is the main contribution to \acs{DD}.\\
\ac{MRI} requires an strong magnetic field. 
This magnetic field is present at all times, due to the need of superconduction. 
Trajectories of moving charged particles such as electrons are affected by the Lorentz-Force first described by \citeauthor{lorentz_versuch_1937} in \citeyear{lorentz_versuch_1937}~\cite{lorentz_versuch_1937}.
Changes in trajectory induce a shift in \acs{DD} towards teh direction of the Lorentz-Force.
Electrons have increased mean free path lengths inside low density mediums such as air.
At the intersection of tissue to air cavities inside the patients anatomie the Lorentz-Force influening the trajectory of electrons causes them to return to the surface of the tissue resulting in an increase in deposited dose.
This occurence is the so called \ac{ERE}~\cite{lee_using_2017} and has significant impact on tumor sides where air cavities are present.


\subsection{Dose Deposition Calculation}

There exist multiple algorithms besides \acs{MC} for \acs{DD} that are used in reasearch aswell as clinically such as collapsed cone~\cite{ahnesjo_collapsed_1989} or pencil beam algorithms~\cite{mohan_differential_1986} which is currently used as a primary dose estimation algorithms in the institutional planning software.
In this study we will focus on the \acs{MC} simulation algorithm because it is used in the secondary dose calculation aswell as dose verification process.\\
As previously mentioned, \acs{IMRT} plans are composed of individual segments for certain irradiation angles determined by the planning algorithm.
Depending on the positions of target volumes aswell as \acs{OAR} and the therefore predefined dose distribution increases the number of segments with increasingly hard to achieve dose limits for \acs{OAR} while maintaining radiation coverage for the target volume. 
Each segment is assinged an individual \acs{MLC} configuration, adapted to the shape of \acs{OAR} and the target volume and a defined number of radiated particles.
The \acs{MLC} of a MR-Linac is composed of 80 leaf pairs of width 7.15~mm in the isocenter resulting in a maximum fieldsize of $57.2~cm \times 22~cm$.
Number of radaited particles is measured in \ac{MU}.
100 \acs{MU} are defined as 1 Gy in depth of 10~cm inside a water phantom with a fieldsize of $10 \times 10~cm$ with a source surface distance of 100~cm.\\
To calculate the \acs{DD} for our training data we utilize the open source software \acs{MC} solution EGSnrc~\cite{noauthor_nrc-cnrcegsnrc_2021} provided under (\citeurl{noauthor_nrc-cnrcegsnrc_2021}).
With the use of EGSnrc we are able to calculate the deposited dose for all segments indivudually. 
Previous work from our institution by \citeauthor{friedel_development_2019} where the use of EGSnrc for the MR-Linac was investigated enabled us to accurately predict the deposited dose for any arbitrary volume.
The output yielded by EGSnrc is normalized to 100~\acs{MU} for the given input parameters such as irradiation volume and accelerator settings.\\

\acs{MC} simulations such as EGSnrc are used for dose calculation in both clinical and research applications.
Photons transfer their energy in the irradiated volume to electrons or positrons in the energy spectrum of interest for external beam radiotherapy. 
In the absence of an analytical solution, without significant simplifications and assumptions for the conditions, the \acs{MC} algorithm is used to simulate a variety of particle trajectories in the desired target volume.
Particle trajectories describe the precise path of a photon from the source to the point in the volume where it has lost all of its energy, including energy transport to secondary electrons in the course of collisions.
The stochastic nature of the interaction processes of photons and electrons requires a large number of simulated particles to obtain an accurate result. 
Long simulation times are the consequence with a number of histories on the order of 10\textsuperscript{7} to 10\textsuperscript{11}.