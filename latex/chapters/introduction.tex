Introduction
\begin{itemize}
    \item \textbf{Genereller Ablauf am MR-Linac stand jetzt}
    \item Person kommt, bekommt ein CT für initiale Planung, danach wird dann bei Bestrahlung des Patienten ein MRT von dem Patienten gemacht. Registrierung von CT und MRT und adaption von contours auf MRT. Dann Online-Plan adaption bei dem Plan aufgrund von adapt to position oder shape. Results in no adaption or adaption of segments shape, monitor units or both. This is repeated for each treatment fraction of the patient (Adaptive radiotherapy: The Elekta Unity MR-linac concept)

    \item \textbf{Wie soll der Ablauf einmal aussehen beim MR-Linac}
    \item Goal is to achieve a MRi-only treatment workflow, including imaging with MRI contouring on MRI and planning and calculation of plans on mri. dose calculation is not possible on mri data because mri is not a quantitative imaging modality, meaning pixel values give no information about the electron density of the underlying tissue or body part. therefore synthetic ct images need to be created from mri images, which enable dose calculations. 
    
    \item \textbf{Und dann auf Dosisdeposition eingehen / Was ist aktuelle Methode} hier noch viel verschieben nach material und methodik
    \item (http://dx.doi.org/10.1118/1.2795842) monte carlo simulations are currently used for dose calculation for radiotreatment plans in a clinical setting. The interactions of photons in human tissue in the energy spectrum of interest for external beam therapy transfer the photon energy on to electrons or positrons. these particles then transfer their energy into the surrounding tissue. before energy deposition the photon and especially the electron undergo a number of elastic and non elsatic interactions with atoms. In the process the main energy loss is caused by inelastic collisions and radiative interactions. the collisions result in ionization and therefore secondary electrons. radative interactions result in a energy trasnfer back to photons. the sum of these phenomena in a photon field results in a coupled electron-photon shower, which can be described by a coupled set of integrodifferntial transport equations. due to the lack of a analystical solution, without any major simplifications and assumptions for conditions, the monte carlo algorithm is used to simululate a multitude of particle histories in the desired target volume. partical histories describe the exact way of a photon from the source to its point in the volume were it has lost all of its energy including enery transport to secondary electrons in the process of collisions. the stochastic nature of the interaction processes of photons and electrons need for a large number of simulated particles to achieve an accurate result of dose deposition. the entirety of all simulated particles then results in an accurate dose distribution which can be used for treatment planning. the need for particle histories in the magnitude of 10\textsuperscript{7} to 10\textsuperscript{11} for accurate dose estimations, result in long simulation times. 

    \item Wo findest Deep Learning Anwendung (Bezug zu Medizin)
    \item Deep learning and especially computer vision is already present in current research of Biology, Physics as well as medicine. there are a multiple fields in which CV can be applied to fields such as dermatology (https://sci-hub.ru /10.1038/nature21056), radiology (https://sci-hub.ru /10.1038/srep24454,
    https://sci-hub.ru /10.1097/rli.0000000000000341), cardiology (https://arxiv.org/abs/1708.09843) or pathology (https://www.nature.com/articles/s41598-020-58467-9.pdf)

    \item Einleitung zu Deep Learning
    \item . deep learning is a prefered tool due to its short inference times aswell as super human performance level on certain tasks. the implementation of the fully convulitional network (https://arxiv.org/pdf/1411.4038.pdf) and its further development of the U-Net utilizing data from higher level representations of the data in the form of skip connections (https://arxiv.org/abs/1505.04597) have revolutionized the application of deep learning for image data and 3d data i the form of a 3D-UNets. 
    
    \item Was ist die Contribution / Aims <- Ziel: Dosisvorhersage mit DL, möglichst robust
    \item In this paper we investiage the capabilities of deep learning in the field of dose predictions for radio treatment plans. we aimed to achieve a robust dose prediction irrespective of the body region of interest and irresprective of the complexity of the treatment plan.
    \item 
\end{itemize}




Show why Radiotherapy is so important: search for sources of application of radiotherapy for different entities. Prostate: \cite{geinitz_3d_2005, nguyen_curative_2005, budiharto_external_nodate} Mamma: \cite{ragaz_adjuvant_1997, lena_combined_nodate, taylor_estimating_2017} Head \& Neck: \cite{datta_head_1990, bhide_advances_2010, castadot_adaptive_2010, morgan_adaptive_2020} Liver: \cite{hoyer_radiotherapy_2012, wulf_stereotactic_2001, wulf_stereotactic_2006, sterzing_stereotactic_2014, witt_mri-guided_2020} Lymph Nodes: \cite{degro_degro_2014, matsushita_stereotactic_2018, mikell_postoperative_2015, lundstedt_long-term_2012, jereczek-fossa_is_2015}

Was ich noch brauche: Infos über MR-Linac, was ist die Vision hinter dem MR Linac (online adaption) 

The use of Magnet Resonace Imaging (MRI) during radiotherapy has opened a variety of new opportunities for treatment optimization. MRI provides a better contrast in soft tissue areas of the body, compared to conventional computed tomograpy (CT), and can be used to assess functional image data from the patient in real time. The enhanched contrast leads to better organs at risk (OAR) and tumor volume delineation. (doi:10.1016/S0360-3016(03)01446-9). Recent research efforts are exploring the capabilities of the hybrid MRI linear accelerator (MRI-Linac) (doi:10.1007/s00066-018-1386-z, doi:10.1016/j.radonc.2007.10.034,  doi:10.1002/acm2.12233). The introduction of the MRI-Linac has transformed the clinical workflow for radiotherapy as well as treatment planning. Patients are required to receive one CT for initial treatment planning. For radiation in each fraction, the inital plan is registrated on the current MRI and optionally adapted to shift or size variation of the tumor volume (doi:10.1016/j.ctro.2019.04.001). Goal is to reach an MRI-only-workflow where image acquisition, treatment planning and radiotherapy only involve the MRI-Linac. To achieve this goal multiple steps in the clinical workflow need to be adapted

behind MRI Linac is an radiotreatment adaption in an onine manner, meaning that a shift of the tumor volume and changes to the patients anatomy due to movement can be considered to adapt the treatmentplan. This results in smaller safety margins (doi:10.1102/1470-7330.2004.0054) for tumor volumes and ultimately result in a lower delivered dose to organs at risk. To achieve this ultimate goal, multiple steps, such as anatomy segmentation, treatmentplan adaption and dose deposition simulations need to be able to be performed in real-time. 

Welche besonderheiten gibt es bei einem MR-Linac im Vergleich zu einem normalem Bestrahler (Stichworte: ERE, Electron Deposition Shift)
Wie funktioniert normale Dosisberechung (Monte Carlo doi:10.1118/1.598917), warum ist der Nutzen davon limitiert wenn man in die online Adaption möchte. 

However, since MC simulation is a stochastic process, the resulting dose map contains inherent quantum noise whose variance is inversely proportional to the number of the simulation histories and, accordingly, to the simulation time. Typically, achieving clinically acceptable precision requires hours of CPU computation time. Graphics processing unit (GPU)-based parallel computation frameworks can accelerate MC simulation to a few minutes for a typical IMRT/VMAT plan (doi:10.1088/0031-9155/55/11/006)

However, several areas in the clinical workflow require real-time dose calculation, such as inverse optimization of the treatment planning process for IMRT and VMAT (doi:10.1088/2632-2153/abdbfe)
especially online radiotherapy and online plan adaption are limited by the time needed to recalculate dose distributions of beam settings and patient anatomies due to moving organs (doi:10.1016/j.clon.2018.08.001)


Machine Learning Teil: Wie wird Machine Learning in verschiedenen bereichen der bestrahlungsplanung bezüglich MRI genutzt:
Eine Implementierung und Nutzung dieser könnte zum Erreichen einer Online-Bestrahlnugsadaption führen

1. Autosegmentation (\cite{kazemifar_segmentation_2018, liang_deep-learning-based_2019}) aswell as uncertrainty (\cite{shen_medical_2019})

2. Radio Treatment Plan optimization (\cite{fan_automatic_2019, liu_deep_2019})

3. Dose Estimation (\cite{kontaxis_deepdose_2020, bai_deep_2021} active denoising of lower history MC Simulations (doi:10.1002/mp.13856 ))

4. Pseudo CT (\cite{han_mr-based_2017, wolterink_deep_2017, dinkla_mr-only_2018})




