Cancer is one of the most contributing diseases leading to death under the age of 70 years in most countries \cite{bray_everincreasing_2021}. 
A multitude of modalities exists for cancer treatment. 
Radiotherapy is a widespread and accessible method throughout the world \cite{shahzad_overview_2018} and is applied to a multitude of tumor entities such as prostate~\cite{geinitz_3d_2005, nguyen_curative_2005, budiharto_external_nodate}, breast~\cite{ragaz_adjuvant_1997, lena_combined_nodate, taylor_estimating_2017}, \ac{HN}~\cite{datta_head_1990, bhide_advances_2010, castadot_adaptive_2010, morgan_adaptive_2020}, liver~\cite{hoyer_radiotherapy_2012, wulf_stereotactic_2001, wulf_stereotactic_2006, sterzing_stereotactic_2014, witt_mri-guided_2020} and \ac{LN} cancer~\cite{degro_degro_2014, matsushita_stereotactic_2018, mikell_postoperative_2015, lundstedt_long-term_2012, jereczek-fossa_is_2015}.
On solid and delimited tumors it is mostly applied after surgical tumor resection.
In a radiotherapy treatment an external irradation source, in the form a linear accelerator is used to radiate the patient.
Tumor cells have decreased repair capabilities compared to healty cells when exposed to ionizing particles, such as photons. 
The application of radiation from photon beams in a fractionated matter results in a destruction of tumor cells, while minimizing the bad side effects on healty tissue.
To archive an optimal radiation dose inside the target volume (tumor cells) and to spare \ac{OAR} the photons are radiated from different angles around the patient. 
There exist multiple application modalities of radiotherapy that depend on the tumor entity, intitutional guidelines and personal preference.
With the \ac{IMRT} treatment modality, which is the most widespread, radiation is applied from multiple specific angles around the patient \cite{cho_intensity-modulated_2018}.
To adapt for different shapes of tumors the shape of the beam can be changed using lead aperatures inside the accelerator head.\\
Individual treatment planning is needed to archive an optimum sparing of \acs{OAR} and tumor dose coverage.
The general workflow for treatment plan creation consists of \ac{CT} aquisition, tumor volume and \ac{OAR} delineation and an iterative process of treatment planning between a medical physicist and an oncologis. 
This entire process can take up to multiple days until a convinient treatment plan is created.\\
A combination of a \ac{MRI} modality and an linear accelerator is currently used at our intitution to enable live imaging during radio treatment. 
Current research effort is going towards the development of the needed software tools to enable an MR-only workflow for radio treatment planning, which makes the \acs{CT} reduntant in the treatment planning pipeline. 
Recent work involves the creation of pseudo CT images \cite{han_mr-based_2017, wolterink_deep_2017, dinkla_mr-only_2018}, delineation of \acs{OAR} \cite{kazemifar_segmentation_2018, liang_deep-learning-based_2019, shen_medical_2019}, plan optimization \cite{fan_automatic_2019, liu_deep_2019}, dose estimation \cite{javaid_mitigating_2019} and .
Ultimately this should lead to a supervised automated real-time treatment plan adation, which enables the treatment plan to be automatically adapted, when the patients anatomy changes between or during fractions.
To enable this protocol multiple steps in the treatment planning pipeline needs to be adapted to be applicable in real-time.
An important part in the pipeline consists of the \ac{DD} calculation, which is needed for the treatment planning software, aswell as dose verficication processes.
State of the art modality for accurate \acs{DD} is a Monte-Carlo simulation \cite{jabbari_review_2011}. 
In this simulation a multitude of particle histories is simulated inside the radiated volume.
This simulation is based on physical processes that are very vell described in literature.
Downside of this process is that, due to the stochastic nature of the \acs{DD} process of particles, a large number of particle histories is needed to yield a dose distribution with sufficient accuracy.
Even in times of parallelization and GPU acceleration techniques this results in long simulation times.
To enable the real time protocol involving the MR-linac, the need for a fast and accurate dose estimation tool is needed.\\
The use of \ac{DL} has shown to be promising in a multitude of applications in \ac{CV} tasks in medicine \cite{esteva_dermatologist-level_2017,cheng_computer-aided_2016,cicero_training_2017,iizuka_deep_2020,poplin_prediction_2018}.
Due to the short inference times when applying \acs{DL} to any given tasks it offers a promising base for a real time application of dose estimation in an MR-only automated treatment plan adaption workflow.\\
The aim of this paper is to improve the state of the art dose estimation tools involving \acs{DL} and evaluate robustness aswell as the generalization capabilities.
We do so by applying a 3D-UNet based \acs{DL} solution to a multitude of tumor entities and treatment plans and evalute its accuracy in dose prediction.