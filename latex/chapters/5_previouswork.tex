
MR guided radiotherapy has received increased attention in the last years, with a multitude of studies and reasearch areas involved.
The problem of fast \acs{DD} estimation is becoming increasingly important with the vision of online and even real-time radio treatment plan adaption.
\acs{DD} algorithms are involved in primary and secondary dose engines aswell as dose verification processes. 
Research contribution goes towards the question of secondary dose engine or dose verficiation, because for both the spcific accelerator parameters are given.\\
Different approaches towards the solution of the deposition involving deep learning have been used. 
\citeauthor{neishabouri_long_2021}~\cite{neishabouri_long_2021} applied a \ac{LSTM} in the application of \acs{DD} for proton radiation.
Input to the \acs{LSTM} network was a sequence of 2D slices from the radiated volume.
Each sequence represents the radiated volume of the proton beam.
Due to the very limited range of protons inside matter and the small beam widening, volumes were small with $15 \times 15 \times 150~voxels$ resulting in a field of view of $30 \times 30 \times 300~mm^2$ with a isotropic resolution of 2~mm per voxel.
Each given slice was therefore of sice $15 \times 15~pixels$.
Network output was the respective dose distribution for the given input slice.
To deviate from patient anatomies they used the approach of creating areas of increased density inside a $15 \times 15 \times 150~voxels$ sized volume.
By variation in position and size of the area the network was able to map \acs{DD} processes to different densities translating to good results when applied to patient anatomies.
Results show 98.57\% gamma passrate (1\%/3mm) for artificial cases and an average gama passrate of 97.85\% for patient test cases.\\
A combination of a 3D-UNet and convolutional \acs{LSTM} networks as skip connections were used by \citeauthor{de_bruijne_high-particle_2021}~\cite{de_bruijne_high-particle_2021}. By aproaching the problem as an active denoising problem, they used a set of noisy monte carlo simulations as an input to the 3D-convLSTM-UNet and the network yielded denoised dose distributions.
Gamma passrates were 94.1\% ± 1.2\% with 3\%/3~mm as the criteria for 5 patient dose distributions.\\
Approaching the problem as an active denoising problem was also done by \citeauthor{neph_deepmcdose_2019}~\cite{neph_deepmcdose_2019} that used 3 input channels consisting of patient CT, photon fluence map and undersampled dose of a single beamlet.
By individual analysis and latter combination of all information they achieved to predict the redisdual dose needed for denoising.
Mean absolute error was reduced to 0.106\% from initial 25.7\% of the undersampled beamlet.
A significant increase in calulation time was also reached with 220~ms compared to the 380~s simulation time needed for a fully samplet beamlet.\\
Inclusion of fluence maps was also pursued by \citeauthor{fan_data-driven_2020}~\cite{fan_data-driven_2020}. By procetion of a 2D fluence map into the 3D patient anatomy they created 3D fluence volume from which a 3D UNet like network make predictions of the deposited dose.
Dose variation normalized to the prescribed dose was 0.17\% ± 2.28\%.\\
\citeauthor{kontaxis_deepdose_2020}~\cite{kontaxis_deepdose_2020} used a 3D-UNet with 5 3D Inputs combing information from the accelerator settings aswell as the radiated volume into the input of the network.
This enables them to predict single segments, which can be added up to enitre treatment plans.
Application was limited to lower abdomen cancer treatment plans with gamma passrates of 99.9 ± 0.3 for (3\%/3mm) for prostate cancer plans.
Dose estimation times were short with approximately 3 minutes for a treatment plan with 41 segments.