
MR guided radiotherapy has received increased attention in the last years, with a multitude of studies and reasearch areas involved.
The problem of fast \acs{DD} estimation is becoming increasingly important with the vision of online and even real-time \acs{RT} plan optimization and adaptation.
\acs{DD} algorithms are involved in primary and secondary dose engines as well as dose verification processes. 
The aim of this work was to replace \acs{MC} simulations with \acs{DL} for secondary dose engine or dose verficiation, because for both the specific accelerator parameters are given.\\
Different approaches towards the solution of the \acs{DD} involving deep learning were pursued in previous work from different working groups. 
\citeauthor{neishabouri_long_2021}~\cite{neishabouri_long_2021} applied a \ac{LSTM} in the application of \acs{DD} for proton irradiation.
Input to the \acs{LSTM} network was a sequence of 2D slices from the irradiated volume.
Each sequence represented the irradiated volume of the proton beam.
Due to the very limited range of protons inside matter and the small beam widening, volumes were small with $15 \times 15 \times 150~voxels$ resulting in a field of view of $30 \times 30 \times 300~mm^2$ with a isotropic resolution of 2~mm per voxel.
Each given slice consisted of size $15 \times 15~pixels$.
Network output was the respective dose distribution for the given input sequence of slices.
To deviate from patient anatomies they used the approach of creating areas of increased density inside a $15 \times 15 \times 150~voxels$ sized volume.
By variation in position and size of the area the network was able to map \acs{DD} processes to different densities translating to good results when applied to patient anatomies.
Results showed 98.57\% mean gamma passrate (1\%/3mm) for artificial cases and an average gama passrate of 97.85\% for patient test cases.\\
A combination of a 3D-UNet and convolutional \acs{LSTM} networks as skip connections were used by \citeauthor{de_bruijne_high-particle_2021}~\cite{de_bruijne_high-particle_2021}.
By approaching the task as an active denoising problem, they used a set of noisy \acs{MC} simulations as an input to the 3D-convLSTM-UNet and the network yielded denoised dose distributions for the given volume.
Gamma passrates were 94.1\% ± 1.2\% with 3\%/3~mm as the gamma criteria for five patient dose distributions.\\
Approaching the problem as an active denoising problem was also done by \citeauthor{neph_deepmcdose_2019}~\cite{neph_deepmcdose_2019} that used three input volumes consisting of patient CT, photon fluence map and undersampled dose of a single beamlet.
By individual analysis of each volume and later combination they achieved to predict the residual dose needed for denoising.
Mean absolute error was reduced to 0.106\% from initial 25.7\% of the undersampled beamlet.
A significant decrease in calulation time was also reached with 220~ms compared to the 380~s for the simulation of a fully sampled beamlet.\\
Inclusion of fluence maps was also pursued by \citeauthor{fan_data-driven_2020}~\cite{fan_data-driven_2020}. 
By projecting a 2D fluence map into 3D space they created 3D fluence volumes from which a 3D UNet like network made predictions of the deposited dose.
Dose variation normalized to the prescribed dose was 0.17\% ± 2.28\%.\\
\citeauthor{kontaxis_deepdose_2020}~\cite{kontaxis_deepdose_2020} used a 3D-UNet with five 3D Inputs to combine information from the accelerator settings as well as the patients anatomy into the input of the network.
This enabled them to predict single segments, which can be added up to entire treatment plans.
Application was limited to lower abdomen cancer treatment plans with gamma passrates of 99.9 ± 0.3 for (3\%/3mm) for prostate cancer plans.
Dose estimation times were short with approximately 3 minutes for a treatment plan with 41 segments.\\

Work of \citeauthor{kontaxis_deepdose_2020}~\cite{kontaxis_deepdose_2020} showed promising results towards a fast \acs{DL} based \acs{DD} estimation tool for radio treatment plans.
We therefore made adjustments to their proposed network to be able to handle the input data from the MR-Linac and evaluated its performance regarding accuracy, robustness and generalization capabilities. 

